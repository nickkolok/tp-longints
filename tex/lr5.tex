\documentclass[a4paper,12pt]{article} %размер бумаги устанавливаем А4, шрифт 12пунктов
\usepackage[T2A]{fontenc}
\usepackage[utf8]{inputenc}
\usepackage[english,russian]{babel} %используем русский и английский языки с переносами
\usepackage{amssymb,amsfonts,amsmath,mathtext,enumerate,float,amsthm} %подключаем нужные пакеты расширений
\usepackage[unicode,colorlinks=true,citecolor=black,linkcolor=black]{hyperref}
%\usepackage[pdftex,unicode,colorlinks=true,linkcolor=blue]{hyperref}
\usepackage{indentfirst} % включить отступ у первого абзаца
\usepackage[dvips]{graphicx} %хотим вставлять рисунки?
\graphicspath{{illustr/}}%путь к рисункам

\makeatletter
\renewcommand{\@biblabel}[1]{#1.} % Заменяем библиографию с квадратных скобок на точку:
\makeatother %Смысл этих трёх строчек мне непонятен, но поверим "Запискам дебианщика"

\usepackage{geometry} % Меняем поля страницы.
\geometry{left=2cm}% левое поле
\geometry{right=1cm}% правое поле
\geometry{top=2cm}% верхнее поле
\geometry{bottom=2cm}% нижнее поле

\renewcommand{\theenumi}{\arabic{enumi}}% Меняем везде перечисления на цифра.цифра
\renewcommand{\labelenumi}{\arabic{enumi}}% Меняем везде перечисления на цифра.цифра
\renewcommand{\theenumii}{.\arabic{enumii}}% Меняем везде перечисления на цифра.цифра
\renewcommand{\labelenumii}{\arabic{enumi}.\arabic{enumii}.}% Меняем везде перечисления на цифра.цифра
\renewcommand{\theenumiii}{.\arabic{enumiii}}% Меняем везде перечисления на цифра.цифра
\renewcommand{\labelenumiii}{\arabic{enumi}.\arabic{enumii}.\arabic{enumiii}.}% Меняем везде перечисления на цифра.цифра

\sloppy


%\renewcommand\normalsize{\fontsize{14}{25.2pt}\selectfont}

\usepackage[backend=biber,style=gost-numeric,sorting=none]{biblatex}
\addbibresource{all.bib}

% Пакет для отображения исходного кода - с http://www.inp.nsk.su/~baldin/LaTeX/lurs-code.pdf
\usepackage{listings}
%\usepackage{listingsutf8}
% подгружаемые языки — подробнее в документации listings
\lstloadlanguages{C++}
% Конфигурируем
\lstset{
	language=C++, % выбираем язык по умолчанию
	frame=single, % рамка
	commentstyle=\itshape\textcolor[rgb]{0.5,0.5,0.5}, % шрифт для комментариев
	stringstyle=\bfseries, % шрифт для строк
	numbers=left,              % где поставить нумерацию строк (слева\справа)
	numberstyle=\tiny,         % размер шрифта для номеров строк
	tabsize=2,                 % размер табуляции по умолчанию равен 2 пробелам
}

% А эта тёмная магия позволяет нормально работать с кириллицей в листингах
% Copyright Nikolay Avdeev aka NickKolok aka Николай Авдеев 2016

% Всем привет из снежного Воронежа!

% This file is part of LISTINGCYR.

%    LISTINGCYR is free software: you can redistribute it and/or modify
%    it under the terms of the GNU General Public License as published by
%    the Free Software Foundation, either version 3 of the License, or
%    (at your option) any later version.

%    LISTINGCYR is distributed in the hope that it will be useful,
%    but WITHOUT ANY WARRANTY; without even the implied warranty of
%    MERCHANTABILITY or FITNESS FOR A PARTICULAR PURPOSE.  See the
%    GNU General Public License for more details.

%    You should have received a copy of the GNU General Public License
%    along with CHAS-CORRECT.  If not, see <http://www.gnu.org/licenses/>.

%  (Этот файл — часть LISTINGCYR.

%   LISTINGCYR - свободная программа: вы можете перераспространять её и/или
%   изменять её на условиях Стандартной общественной лицензии GNU в том виде,
%   в каком она была опубликована Фондом свободного программного обеспечения;
%   либо версии 3 лицензии, либо (по вашему выбору) любой более поздней
%   версии.

%   LISTINGCYR распространяется в надежде, что она будет полезной,
%   но БЕЗО ВСЯКИХ ГАРАНТИЙ; даже без неявной гарантии ТОВАРНОГО ВИДА
%   или ПРИГОДНОСТИ ДЛЯ ОПРЕДЕЛЕННЫХ ЦЕЛЕЙ. Подробнее см. в Стандартной
%   общественной лицензии GNU.

%   Вы должны были получить копию Стандартной общественной лицензии GNU
%   вместе с этой программой. Если это не так, см.
%   <http://www.gnu.org/licenses/>.)
%





% Юзер, помни!
% Сей файл под GNU GPLv3.
% Слинковался - открой сорцы!
% Copyright Nikolay Avdeev 2016
% nickkolok@mail.ru or avdeev@math.vsu.ru

% Пользуясь случаем, передаю привет из Воронежа товарищу @virens

%Спасибо юзеру waverider за http://dxdy.ru/topic18924-15.html
\lstset{literate=
	{А}{{\CYRA}}1
	{Б}{{\CYRB}}1
	{В}{{\CYRV}}1
	{Г}{{\CYRG}}1
	{Д}{{\CYRD}}1
	{Е}{{\CYRE}}1
	{Ё}{{\CYRYO}}1
	{Ж}{{\CYRZH}}1
	{З}{{\CYRZ}}1
	{И}{{\CYRI}}1
	{Й}{{\CYRISHRT}}1
	{К}{{\CYRK}}1
	{Л}{{\CYRL}}1
	{М}{{\CYRM}}1
	{Н}{{\CYRN}}1
	{О}{{\CYRO}}1
	{П}{{\CYRP}}1
	{Р}{{\CYRR}}1
	{С}{{\CYRS}}1
	{Т}{{\CYRT}}1
	{У}{{\CYRU}}1
	{Ф}{{\CYRF}}1
	{Х}{{\CYRH}}1
	{Ц}{{\CYRC}}1
	{Ч}{{\CYRCH}}1
	{Ш}{{\CYRSH}}1
	{Щ}{{\CYRSHCH}}1
	{Ъ}{{\CYRHRDSN}}1
	{Ы}{{\CYRERY}}1
	{Ь}{{\CYRSFTSN}}1
	{Э}{{\CYREREV}}1
	{Ю}{{\CYRYU}}1
	{Я}{{\CYRYA}}1
	{а}{{\cyra}}1
	{б}{{\cyrb}}1
	{в}{{\cyrv}}1
	{г}{{\cyrg}}1
	{д}{{\cyrd}}1
	{е}{{\cyre}}1
	{ё}{{\cyryo}}1
	{ж}{{\cyrzh}}1
	{з}{{\cyrz}}1
	{и}{{\cyri}}1
	{й}{{\cyrishrt}}1
	{к}{{\cyrk}}1
	{л}{{\cyrl}}1
	{м}{{\cyrm}}1
	{н}{{\cyrn}}1
	{о}{{\cyro}}1
	{п}{{\cyrp}}1
	{р}{{\cyrr}}1
	{с}{{\cyrs}}1
	{т}{{\cyrt}}1
	{у}{{\cyru}}1
	{ф}{{\cyrf}}1
	{х}{{\cyrh}}1
	{ц}{{\cyrc}}1
	{ч}{{\cyrch}}1
	{ш}{{\cyrsh}}1
	{щ}{{\cyrshch}}1
	{ъ}{{\cyrhrdsn}}1
	{ы}{{\cyrery}}1
	{ь}{{\cyrsftsn}}1
	{э}{{\cyrerev}}1
	{ю}{{\cyryu}}1
	{я}{{\cyrya}}1
}

% Люди, любите друг друга, используйте Linux и поступайте на матфак ВГУ!








\begin{document}

\setcounter{page}{2}

\section*{Предварительные сведения}

В качестве основного пособия по языку C++ используется \cite{chmyhalo}.


Блок-схемы выполняются в соответствии с \cite{gost-block-scheme};
ряд примеров имеется в \cite{wiki-block-scheme}.


\section*{Постановка задачи}
Составить алгоритм и написать программу
для выполнения следующих действий в кольце вычетов:
сложение, вычитание, умножение, возведение в степень,
нахождение обратного элемента.


\section*{Организация работы с длинными числами}
Для сложения и вычитания в кольце вычетов используются только аддитивные операции,
поэтому в качестве основания системы счисления используется $10^9$.

Умножение, возведение в степень и отыскание обратного элемента требуют
мультипликативных операций,
поэтому в качестве основания системы счисления используется $10^4$.

Условимся, что функция обращения элемента возвращает 0,
если элемент необратим в данном кольце.
Это не приведёт к путанице, поскольку 0 не является обратным ни к какому элементу,
и вышестоящий алгоритм, получив 0, может корректно обработать случай необратимости.

\section*{Используемые алгоритмы}

Сложение и вычитание в кольце,
а также нахождение обратного элемента через коэффициенты линейного представления НОД
представляются тривиальными (см. рис. \ref{bs_plus}, \ref{bs_minus}, \ref{bs_reverse}).

Умножение и возведение в степень реализуются с применением алгоритма Монтгомери~\cite{Glukhov}.
Для умножения используется алгоритм 1.1 из~\cite{Glukhov} с незначительной модицикацией:
в некоторых случаях вычисляются не все цифры промежуточного произведения (см. рис. \ref{bs_lr5_multiplyDigits}).
Алгоритм 1.2 реализовать не удалось, по итогам ручной трассировки с исходными данными
$N=1000000001$, $\beta=10^4$, $x=37000$, $y=60000$ можно предположить,
что в алгоритме ошибка (но не исключается и ошибка при трассировке).

Умножение по Монтгомери реализовано в виде двух вложенных друг в друга вызовов функции $\varphi_R$
(см.~\cite{Glukhov}).
Функция createMontgomery(...) по заданному $N$
восстанавливает (через передачу по ссылке) значения $R$, $N'$ (переменная Ns) и $w = R^2 \mod N$
(см. рис. \ref{bs_lr5_createMontgomery}).
Это обеспечивает возможность повторного использования данных значений.

Возведение в степень также использует идеи из~\cite{Glukhov},
включая бинарный алгоритм, однако выполняется с учётом того,
что извлечение младшего бинарного разряда при текущей реализации
работы с длинными числами удобнее, чем извлечение старшего.

В случаях, когда умножение~--- а, следовательно, и возведение в степень~--- по Монтгомери~\cite{Glukhov}
невозможны (т.е. когда модуль кольца не взаимно просто с основанием системы счисления),
будет необходимо выполнять умножение целых чисел без учёта кольца с последующим вычислением остатка
(или, соответственно, возведение в степень).
При умножении в кольце не по Монтгомери затруднительно предложить оптимизацию (см. рис. \ref{bs_multiply_direct});
с возведением в степень дела обстоят иначе (см. рис. \ref{bs_power_direct}).



\section*{Исходный код программы}

\lstinputlisting[caption={class\_int.cpp}]{../class_int.cpp}
\lstinputlisting[caption={class\_int\_square.cpp}]{../class_int_square.cpp}
\lstinputlisting[label=BigIntClass,caption={class\_int\_division.cpp}]{../class_int_division.cpp}
\lstinputlisting[caption={class\_int\_gcd.cpp}]{../class_int_gcd.cpp}
\lstinputlisting[caption={class\_int\_ring.cpp}]{../class_int_ring.cpp}
\lstinputlisting[caption={ring\_reverse.cpp}]{../ring_reverse.cpp}
\lstinputlisting[caption={ring\_pm.cpp}]{../ring_pm.cpp}
\lstinputlisting[caption={ring\_mult.cpp}]{../ring_mult.cpp}

\section*{Блок-схемы}

\begin{figure}[ht]
	\centering
	\includegraphics[width=0.6\textwidth]{lr5_sumRing.pdf}
	\caption{
		Блок-схема алгоритма нахождения суммы элементов в кольце вычетов:
		функция BigInt ringReverse(BigInt modulo, BigInt x)
	}
	\label{bs_plus}
\end{figure}

\begin{figure}[ht]
	\centering
	\includegraphics[width=0.6\textwidth]{lr5_subtractRing.pdf}
	\caption{
		Блок-схема алгоритма нахождения разности элементов в кольце вычетов:
		функция BigInt ringReverse(BigInt modulo, BigInt x)
	}
	\label{bs_minus}
\end{figure}

\begin{figure}[ht]
	\centering
	\includegraphics[width=0.6\textwidth]{lr5_ringReverse.pdf}
	\caption{
		Блок-схема алгоритма нахождения обратного элемента в кольце вычетов:
		функция BigInt ringReverse(BigInt modulo, BigInt x)
	}
	\label{bs_reverse}
\end{figure}

\begin{figure}[ht]
	\centering
	\includegraphics[width=0.32\textwidth]{lr5_multiplyRingDirect.pdf}
	\caption{
		Блок-схема алгоритма непосредственного умножения элементов в кольце вычетов:
		функция BigInt multiplyRingDirect(BigInt x, BigInt y, BigInt modulo)
	}
	\label{bs_multiply_direct}
\end{figure}

\begin{figure}[ht]
	\centering
	\includegraphics[width=0.64\textwidth]{lr5_raiseToPowerRingDirect.pdf}
	\caption{
		Блок-схема алгоритма непосредственного возведения в степень в кольце вычетов:
		функция BigInt multiplyRingDirect(BigInt x, BigInt y, BigInt modulo)
	}
	\label{bs_power_direct}
\end{figure}

\begin{figure}[ht]
	\centering
	\includegraphics[width=\textwidth]{lr5_multiplyDigits.pdf}
	\caption{
		Блок-схема алгоритма вычисления последних цифр произведения:
		функция BigInt multiplyDigits(BigInt x1, BigInt x2, int digits)
	}
	\label{bs_lr5_multiplyDigits}
\end{figure}

\begin{figure}[ht]
	\centering
	\includegraphics[width=0.34\textwidth]{lr5_createMontgomery.pdf}
	\caption{
		Блок-схема алгоритма вычисления вспомогательных значений для алгоритма Монтгомери:
		функция void createMontgomery(BigInt N, BigInt\& R, BigInt\& Ns, BigInt\& w)
	}
	\label{bs_lr5_createMontgomery}
\end{figure}

\begin{figure}[ht]
	\centering
	\includegraphics[width=0.64\textwidth]{lr5_phiMontgomery.pdf}
	\caption{
		Блок-схема алгоритма вычисления $\varphi_R$ для алгоритма Монтгомери:
		функция BigInt phiMontgomery(BigInt x, BigInt y, BigInt\& N, BigInt\& R, BigInt\& Ns)
	}
	\label{bs_lr5_phiMontgomery}
\end{figure}

\begin{figure}[ht]
	\centering
	\includegraphics[width=0.64\textwidth]{lr5_multiplyRing.pdf}
	\caption{
		Блок-схема алгоритма умножения в кольце: алгоритма Монтгомери или непосредственного умножения:
		функция BigInt multiplyRing(BigInt x, BigInt y, BigInt N)
	}
	\label{bs_lr5_multiplyRing}
\end{figure}

\begin{figure}[ht]
	\centering
	\includegraphics[width=0.64\textwidth]{lr5_powerRingMontgomery.pdf}
	\caption{
		Блок-схема алгоритма возведения в степень в кольце: алгоритма Монтгомери или непосредственного возведения в степень:
		функция BigInt powerRingMontgomery(BigInt x, BigInt y, BigInt\& N, BigInt\& R, BigInt\& Ns, BigInt\& w)
	}
	\label{bs_lr5_powerRingMontgomery}
\end{figure}

\begin{figure}[ht]
	\centering
	\includegraphics[width=0.64\textwidth]{lr5_powerRing.pdf}
	\caption{
		Блок-схема алгоритма возведения в степень в кольце: алгоритма Монтгомери или непосредственного возведения в степень:
		функция BigInt powerRing(BigInt x, BigInt y, BigInt N)
	}
	\label{bs_lr5_powerRing}
\end{figure}

\clearpage

\section*{Результаты работы программы}

\subsection*{Нахождение обратного элемента. Запуск №1}
\begin{verbatim}
Введите модуль кольца:
37
Введите обращаемый элемент:
3
Обратный элемент:
25


------------------
(program exited with code: 0)
Press return to continue
\end{verbatim}

\subsection*{Нахождение обратного элемента. Запуск №2}
\begin{verbatim}
Введите модуль кольца:
100
Введите обращаемый элемент:
25
Обратного элемента не существует


------------------
(program exited with code: 0)
Press return to continue
\end{verbatim}

\subsection*{Нахождение обратного элемента. Запуск №3}
\begin{verbatim}
Введите модуль кольца:
1000000000000000001
Введите обращаемый элемент:
222222222222222222
Обратный элемент:
249999999999999998


------------------
(program exited with code: 0)
Press return to continue
\end{verbatim}

\subsection*{Нахождение обратного элемента. Запуск №4}
\begin{verbatim}
Введите модуль кольца:
100000000000000000
Введите обращаемый элемент:
50000000
Обратного элемента не существует


------------------
(program exited with code: 0)
Press return to continue
\end{verbatim}

\subsection*{Сложение. Запуск №1}
\begin{verbatim}
Введите модуль кольца вычетов:
17
Введите первый операнд:
11
Введите второй операнд:
3
Введите знак операции:  +  или  -
+
14


------------------
(program exited with code: 0)
Press return to continue
\end{verbatim}

\subsection*{Сложение. Запуск №2}
\begin{verbatim}
Введите модуль кольца вычетов:
17
Введите первый операнд:
11
Введите второй операнд:
12
Введите знак операции:  +  или  -
+
6


------------------
(program exited with code: 0)
Press return to continue
\end{verbatim}

\subsection*{Сложение. Запуск №3}
\begin{verbatim}
Введите модуль кольца вычетов:
100000000000000000000001
Введите первый операнд:
2222222222222222222222
Введите второй операнд:
3333333333333333333333
Введите знак операции:  +  или  -
+
5555555555555555555555


------------------
(program exited with code: 0)
Press return to continue
\end{verbatim}

\subsection*{Сложение. Запуск №4}
\begin{verbatim}
Введите модуль кольца вычетов:
1000000000000000000001
Введите первый операнд:
222222222222222222222
Введите второй операнд:
999999999999999999999
Введите знак операции:  +  или  -
+
222222222222222222220


------------------
(program exited with code: 0)
Press return to continue
\end{verbatim}


\subsection*{Вычитание. Запуск №1}
\begin{verbatim}
Введите модуль кольца вычетов:
100000000000000001
Введите первый операнд:
22222222222222222
Введите второй операнд:
11111111111111111
Введите знак операции:  +  или  -
-
11111111111111111


------------------
(program exited with code: 0)
Press return to continue
\end{verbatim}

\subsection*{Вычитание. Запуск №2}
\begin{verbatim}
Введите модуль кольца вычетов:
10000000000000000000000002
Введите первый операнд:
2222222222222222222222223
Введите второй операнд:
9999999999999999999999999
Введите знак операции:  +  или  -
-
2222222222222222222222226


------------------
(program exited with code: 0)
Press return to continue
\end{verbatim}

\subsection*{Умножение (алгоритм Монтгомери применим)}
\begin{verbatim}
Введите модуль кольца вычетов:
1000000001
Введите первый операнд:
37000
Введитe второй операнд:
6000000
Введите знак операции:  *  или  ^
*
Результат:
999999779


------------------
(program exited with code: 0)
Press return to continue
\end{verbatim}

\subsection*{Умножение (алгоритм Монтгомери неприменим)}
\begin{verbatim}
Введите модуль кольца вычетов:
10000000000000002
Введите первый операнд:
370000000000
Введитe второй операнд:
60000000000
Введите знак операции:  *  или  ^
*
Результат:
9999999995560002


------------------
(program exited with code: 0)
Press return to continue
\end{verbatim}

\subsection*{Возведение в степень (алгоритм Монтгомери применим)}
\begin{verbatim}
Введите модуль кольца вычетов:
10000003
Введите первый операнд:
370
Введитe второй операнд:
56
Введите знак операции:  *  или  ^
^
Результат:
5221479


------------------
(program exited with code: 0)
Press return to continue
\end{verbatim}

\subsection*{Возведение в степень (алгоритм Монтгомери неприменим)}
\begin{verbatim}
Введите модуль кольца вычетов:
10020
Введите первый операнд:
370
Введитe второй операнд:
33
Введите знак операции:  *  или  ^
^
Результат:
2860


------------------
(program exited with code: 0)
Press return to continue
\end{verbatim}


\printbibliography
\end{document}
